%%%%%%%%%%%%%%%%%%%%%%%%%%%%%%%%%%%%%%%%%
% Sullivan Business Report
% LaTeX Template
% Version 1.0 (May 5, 2022)
%
% This template originates from:
% https://www.LaTeXTemplates.com
%
% Author:
% Vel (vel@latextemplates.com)
%
% License:
% CC BY-NC-SA 4.0 (https://creativecommons.org/licenses/by-nc-sa/4.0/)
%
%%%%%%%%%%%%%%%%%%%%%%%%%%%%%%%%%%%%%%%%%

%----------------------------------------------------------------------------------------
%	CLASS, PACKAGES AND OTHER DOCUMENT CONFIGURATIONS
%----------------------------------------------------------------------------------------

\documentclass[
	a4paper, % Paper size, use either a4paper or letterpaper
	12pt, % Default font size, the template is designed to look good at 12pt so it's best not to change this
	%unnumberedsections, % Uncomment for no section numbering
]{CSSullivanBusinessReport}

%\addbibresource{sample.bib} % BibLaTeX bibliography file

%----------------------------------------------------------------------------------------
%	REPORT INFORMATION
%----------------------------------------------------------------------------------------

\reporttitle{SafeSign} % The report title to appear on the title page and page headers, do not create manual new lines here as this will carry over to page headers

\reportsubtitle{Sistema empotrado de establecimiento de \\ velocidad máxima 
con respecto a las condiciones meteorológicas} % Report subtitle, include new lines if needed

\reportauthors{José Alfredo Lopez José, \\ Jesús Carrascosa Carro, \\ José Pablo Ruiz Pérez, \\ Yeray Rincón Cardoso } % Report authors/group/department, include new lines if needed
%TODO: cambiar si fuera nencesario
\reportdate{2024-03-04} % Report date, include new lines for additional information if needed

\rightheadercontent{\includegraphics[width=3cm]{Images/LinterniTek.png}} % The content in the right header, you may want to add your own company logo or use your company/department name or leave this command empty for no right header content
\renewcommand{\contentsname}{Índice}

%----------------------------------------------------------------------------------------

\begin{document}

%----------------------------------------------------------------------------------------
%	TITLE PAGE
%----------------------------------------------------------------------------------------

\thispagestyle{empty} % Suppress headers and footers on this page

\begin{fullwidth} % Use the whole page width
	\vspace*{-0.075\textheight} % Pull logo into the top margin
	
	\hfill\includegraphics[width=5cm]{Images/LinterniTek.png} % Company logo

	\vspace{0.15\textheight} % Vertical whitespace

	\parbox{0.9\fulltextwidth}{\fontsize{50pt}{52pt}\selectfont\raggedright\textbf{\reporttitle}\par} % Report title, intentionally at less than full width for nice wrapping. Adjust the width of the \parbox and the font size as needed for your title to look good.
	
	\vspace{0.03\textheight} % Vertical whitespace
	
	{\LARGE\textit{\textbf{\reportsubtitle}}\par} % Subtitle
	
	\vfill % Vertical whitespace
	
	{\Large\reportauthors\par} % Report authors, group or department
	
	\vfill\vfill\vfill % Vertical whitespace
	
	{\large\reportdate\par} % Report date
\end{fullwidth}

\newpage

%----------------------------------------------------------------------------------------
%	DISCLAIMER/COPYRIGHT PAGE
%----------------------------------------------------------------------------------------

\thispagestyle{empty} % Suppress headers and footers on this page

\begin{twothirdswidth} % Content in this environment to be at two-thirds of the whole page width
	\footnotesize % Reduce font size
	
	
	\subsection*{Copyright}
	
	\textcopyright~[2024] [LinterniTek S.Coop.And] 
	
	%TODO el copy
	\subsection*{Información de Contacto}
	% TODO Poner info de contacto
	E.T.S. De Ing. Informática\\
	Av. Reina mercedes S/N\\
    Sevilla, España\\
	41005\\
	
	NIF: F-1234567-E
	
	Correo electrónico: \href{mailto:contacto@linternitek.com}{contacto@linternitek.com}.
	
	\vfill % Push the following down to the bottom of the page
	\subsection*{Acerca del documento}
	\subsubsection*{Registro de Cambios}
	
	\scriptsize % Reduce font size further
	
	\begin{tabular}{@{} L{0.05\linewidth} L{0.15\linewidth} L{0.6\linewidth} @{}} % Column widths specified here, change as needed for your content
		\toprule % Modificar
		v1.0 & 2024-02-05 & Creada la plantilla \\
		v1.1 & 2024-02-27 &  Pequeños cambios en el contexto\\ 
		v1.2 & 2024-03-01 & Modificada la información de contacto\\
		\bottomrule
	\end{tabular}
    \subsubsection*{Estado del documento}
    \scriptsize
      
        \begin{tabular}{cc}
        \toprule
            Código& PGPI-LTK-SFSNG-MANDATO \\
             Estado& APROBADO\\
             Fecha& 2024-03-04 \\
             Versíon& 1.2\\
             \bottomrule
        \end{tabular}
       
    \subsubsection*{Distribución del documento}
        \begin{tabular}{|c|c|c|c|}
        \toprule
         \textbf{Acción} & \textbf{Nombre} &\textbf{Firma} & \textbf{Fecha}\\
         \midrule
         ELABORACIÓN & Jesús Carrascosa Carro & \textit{yisus }& 2024-03-01\\
         \midrule
         REVISIÓN & Jesús Carrascosa Carro & \textit{yisus} & 2024-03-02\\
         REVISIÓN & José Alfredo Lopez José & \textit{JoseAl} & 2024-03-02\\
         \midrule
         APROBACIÓN & Jesús Carrascosa Carro & yisus & 2024-03-04\\
         APROBACIÓN & José Alfredo Lopez José & \textit{JoseAl} & 2024-03-04\\
         APROBACIÓN & José Pablo Ruiz Pérez & \textit{RuizPerez} & 2024-03-04\\
         APROBACIÓN & Yeray Rincón Cardoso & \textit{yeray} & 2024-03-04\\
         \bottomrule
         
        \end{tabular}
\end{twothirdswidth}

\newpage

%----------------------------------------------------------------------------------------
%	TABLE OF CONTENTS
%----------------------------------------------------------------------------------------

\begin{twothirdswidth} % Content in this environment to be at two-thirds of the whole page width
	\tableofcontents % Output the table of contents, automatically generated from the section commands used in the document
\end{twothirdswidth}

\newpage

%----------------------------------------------------------------------------------------
%	SECTIONS
%----------------------------------------------------------------------------------------
\begin{fullwidth}
    

\section{Contexto}
\subsection{Sobre Nosotros}
Nuestra empresa, LinterniTek, se dedica al desarrollo de Sistemas Empotrados aplicados al ámbito de la automoción y la seguridad vial, con la misión de optimizar y mejorar la seguridad vial y la gestión del tráfico en tiempo real utilizando sistemas autónomos y conectados.  
\par
Esta propuesta de proyecto informático forma parte de la visión de la empresa de crear soluciones innovadoras para gestionar y adaptar las condiciones de las carreteras a las necesidades de los usuarios, de manera que se reduzca el riesgo de accidentes de tráfico y se mejore la fluidez del tráfico
\subsection{Sobre el problema }
En los últimos años en España, debido a las fluctuaciones extremas y rápidas del clima (que en caso de precipitaciones y neblina se traduce en una reducción de la visibilidad) y sumadas a la deficiente señalización, el estado de deterioro de algunas carreteras y la influencia de accidentes cercanos recientes lo que dificulta que los conductores puedan evaluar su entorno de manera adecuada, lo que a su vez lleva a comportamientos impredecibles. Esto ha llevado a un aumento en los accidentes de tráfico. Esta problemática se agravará a medida que el creciente cambio climático provoque una mayor impredecibilidad en las condiciones temporales, requiriendo una mayor capacidad de adaptación con menores márgenes de reacción. 
\par 
Con el fin de ilustrar la problemática, a continuación, se detallan diversos accidentes cuya causa principalmente fue la baja visibilidad: 
\begin{itemize}
    \item El martes 20 de miércoles 20024, sobre las 4 de la madrugada, ocurrió un fatal accidente en la autopista AP-4 que conecta Cádiz con Sevilla. Debido a la baja visibilidad y a una mala señalización, un camión (que además iba por encima del límite de velocidad permitido por el camión y por la autopista) embistió un control de la Guardia Civil situado en la salida a Utrera en sentido Sevilla, matando a seis personas en el suceso.

    \item Unos días después, a la altura de la salida hacia el aeropuerto de Jerez y en sentido Sevilla, ocurrió otro accidente en el cual, debido a la lluvia intensa y a la velocidad que dicho conductor llevaba en el momento (más de 120 kilómetros por hora), acabó atravesando la mediana y casi llegando al carril contrario de la circulación.
\end{itemize}
\end{fullwidth}
\newpage
\begin{fullwidth}
    

\section{Alcance del proyecto}
\subsection{Objetivos}
El objetivo principal de este proyecto es llevar a cabo un sistema de señalización automático de la velocidad (denominado SafeSign) que tenga en cuenta diversos factores como pueden ser ambientales, lumínicos, tráfico, accidentales e incluso estadísticos para poder determinar una nueva velocidad máxima permitida y así avise a los conductores de los posibles peligros y de la nueva velocidad permitida mediante señales electrónicas dispuestas en puntos estratégicos de la carretera. Pretendiendo así lo siguiente:  
\begin{itemize}
    \item Poder asegurar que los usuarios puedan conducir de manera segura por una carretera independientemente de las condiciones en la que se encuentre la misma. 

    \item Proporcionar información a los usuarios que usen dicha carretera del estado en el que se encuentra.

    \item Realizar un control automático sobre la velocidad de las carreteras teniendo en cuenta los factores meteorológicos que haya. 

    \item  Reducir el número de accidentes de tráfico y víctimas que puedan ocasionarse debido a las condiciones meteorológicas. 

    \item  Generar y proporcionar un corpus de datos que permitan llevar a cabo evaluaciones estadísticas y análisis a gran escala con Big Data e Inteligencia Artificial. 

    \item  Comercializar este sistema de forma internacional para así poder reducir a nivel mundial el número de accidentes de tráfico. 
\end{itemize}



\subsection{Alcance}
\subsubsection{Organizativo}
Al no ser un proyecto de despliegue, sino de elaboración de un dispositivo, no es necesario involucrar a los organismos encargados de la infraestructura de la vigilancia y control del tráfico(En el caso de España, serían la Gerencia de Informática y la Subdirección General de Gestión de la Movilidad y Tecnología, ambas pertenecientes a la Dirección General de Tráfico) más allá de una posible financiación u asesoramiento. Sin embargo, con los recursos humanos qué dispone LinterniTek, solo es posible realizar \textit{in-house} el diseño y construcción del dispositivo (en lo que a hardware y software), siendo necesario  subcontratar los siguientes servicios: 
\begin{itemize}
    \item \textbf{Asesoramiento legal:} Para la correcta gestión e interpretación de patentes, homologaciones, normativas \ldots
    \item \textbf{Marketing:} Para poder promocionarnos de cara a la captación de fondos o una comercialización de producto.
    \item \textbf{Traducción e Interpretación:} De cara a traducir y adaptar la documentación tanto interna y externa del proyecto como el material promocional.

    \item \textbf{Fabricación de PCBs:} Autoexplicativo.
\end{itemize}
\begin{itemize}
    \item \textbf{Asesoramiento Legal}
\end{itemize}
\subsubsection{Funcional}
 Los objetivos funcionales que debe cumplir el proyecto son los siguientes: 
  \begin{itemize}
      \item \textbf{Hardware:} Se ha de desarrollar un dispositivo electrónico que integre múltiples sensores meteorológicos(pluviómetro) junto con sensores de luminosidad, e infrarrojos (con la finalidad de detectar la presencia de coches), junto con la correspondiente señal electrónica homologada. Además de contar con múltiples canales de comunicación como redes celulares(5g) y de larga distancia(LoraWan) o mesh (ZigBee). Todo esto se presenta de forma intencionadamente ambigua, ya que las homologaciones, certificaciones, normativas y estándares que sean necesarios cumplimentar varían según la zona donde se vaya a implantar.
      \item \textbf{Software:}
      \item \textbf{ API:} Ha de poseer las siguientes funciones: 
      \begin{itemize}
          \item   Enviar los datos meteorológicos recopilados a un servidor central para su posterior análisis.
          \item Informar y enviar en tiempo real la velocidad máxima establecida a un servidor central.
          \item Informar y enviar información relativa al estado de los sensores y del microcontrolador a un servidor central.
          \item Permitir el \textit{override} de la velocidad máxima permitida por parte de un servidor central. Esto permitiría la coordinación de múltiples señales SafeSign mediante un servidor central. O bien, en caso de mal funcionamiento, establecer una velocidad máxima preventiva o la deshabilitación de la señal.
          \item Permitir las actualizaciones del firmware de forma remota.
          
      \end{itemize}
      
  \end{itemize}
\subsubsection{Temporal}
El proyecto se plantea con una duración máxima de 2 años sin tener en cuenta el tiempo necesario 
\subsection{Restricciones,Riesgos y Asunciones}
\subsection{Dependencias}
Este proyecto no depende de ningún proyecto previo.

\end{fullwidth}


%----------------------------------------------------------------------------------------

\end{document}
